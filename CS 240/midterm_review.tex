\documentclass[11pt, oneside]{article}
\usepackage[margin=1in]{geometry}
\geometry{letterpaper}
\usepackage{amssymb}
\usepackage[fleqn]{amsmath}
\usepackage[sharp]{easylist}
\usepackage{relsize}

\pagenumbering{gobble}              % No page numbering
\setlength{\parindent}{0em}         % No paragraph indenting
\setlength{\parskip}{0.5em}         % Paragraph spacing

\newcommand*{\begineasylist}{\begin{easylist}[itemize]\ListProperties(Style*=$\bullet$\quad, Style2*=\tiny$\blacksquare$\quad, Style3*=$\circ$\quad, Style4*=$\diamond$\quad, FinalSpace=1em, Space=0em, Space*=0em)}

\newcommand*{\begineasylistnumbered}{\begin{easylist}[enumerate]\ListProperties(Numbers=a, Space=0em, Space*=0em)}

\begin{document}

\section*{CS 240 Midterm Review (Module 1--7)}

\subsection*{Asymptotic Analysis}
\begineasylist

# Problem instance (I) -- \emph{input} for the specified problem
# Problem solution -- \emph{output} for the specified problem instance
# Problem size -- Size(I) = size of instance I

# Algorithm - a step-by-step process for carrying out a series of computations
## An algorithm A solves a problem P if, for every instance I of P, A computes a valid solution for I in \underline{finite} time

# RAM model
## Assume any memory access \& primitive operation is constant time
## Assume infinite amount of memory
## Sequential operation
## Running time is determined by the \# of memory accesses \& primitive operations

# Order notations
## \underline{$f(n) \in O(g(n))$} if $\exists \ c > 0$ and $n_0 > 0$ such that $0 \leq f(n) \leq cg(n) \ \forall \ n \geq n_0$
### $f$ ``grows no faster than'' $g$
### $f$ is ``upper-bounded'' by $g$ ($\leq$)

## \underline{$f(n) \in \Omega(g(n))$} if $\exists \ c > 0$ and $n_0 > 0$ such that $0 \leq cg(n) \leq f(n) \ \forall \ n \geq n_0$
### $f$ ``grows no slower than'' $g$
### $f$ is ``lower-bounded'' by $g$ ($\geq$)

## \underline{$f(n) \in \Theta(g(n))$} if $\exists \ c_1, c_2 > 0$ and $n_0 > 0$ such that $0 \leq c_1g(n) \leq f(n) \leq c_2g(n) \ \forall \ n \geq n_0$
### $f$ and $g$ grow at the same rate

## \underline{$f(n) \in o(g(n))$} if $\forall \ c > 0, \exists \ n_0 > 0$ such that $0 \leq f(n) < cg(n) \ \forall \ n \geq n_0$
### $f$ is ``\emph{strictly} upper-bounded'' by $g$ ($<$)

## \underline{$f(n) \in \omega(g(n))$} if $\forall \ c > 0, \exists \ n_0 > 0$ such that $0 \leq cg(n) < f(n) \ \forall \ n \geq n_0$  
### $f$ is ``\emph{strictly} lower-bounded'' by $g$ ($>$)

## Suppose $L = \lim_{n \rightarrow \infty} \dfrac{f(n)}{g(n)}$
### If $L = 0$ then $f \in o(g)$
### If $0 < L < \infty$ then $f \in \Theta(g)$
### If $L = \infty$ then $f \in \omega(g)$

## If $f \in O(g)$ and $f \in \Omega(g)$, then $f \in \Theta(g)$

# Loop analysis
## Begin from the innermost nested loop; use $\sum$ for each outer loop

# Recurrence relations analysis
## e.g. mergesort:
## Step 1: split array of length $n$ into two subarrays, of lengths $\lceil \frac{n}{2} \rceil$ and $\lfloor \frac{n}{2} \rfloor$ ($T = \Theta(n)$)
## Step 2: recursively run mergesort on subarrays ($T = T(\lceil \frac{n}{2} \rceil) + T(\lfloor \frac{n}{2} \rfloor)$)
## Step 3: merge sorted subarrays into a single sorted array ($T = \Theta(n)$)
## Thus the \underline{recurrence relation} is
\begin{align*}
T(n) &= \Theta(1) &\text{ if } n = 1\\
T(n) &= T(\lceil \frac{n}{2} \rceil) + T(\lfloor \frac{n}{2} \rfloor) + \Theta(n) &\text{ if } n > 1\\
&= 2T(\frac{n}{2}) + cn\\
&= 2(2T(\frac{n}{4}) + \frac{cn}{2}) + cn\\
&= \ldots\\
&= 2^kT(\frac{n}{2^k}) + kcn &\text{where } k = \log n\\
&= nT(1) + \log n (cn)\\
&\in \Theta(n \log n)
\end{align*}

## \textbf{In general}, $\{T(n) = T(n/2) + c\} \in \Theta(n \log n)$

\end{easylist}
\subsection*{Priorty Queues}
\begineasylist

#

\end{easylist}

\end{document}