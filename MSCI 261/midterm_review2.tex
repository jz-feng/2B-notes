\documentclass[11pt, oneside]{article}
\usepackage[margin=1in]{geometry}
\geometry{letterpaper}
\usepackage{amssymb}
\usepackage[fleqn]{amsmath}
\usepackage[sharp]{easylist}
\usepackage{relsize}

\pagenumbering{gobble}              % No page numbering
\setlength{\parindent}{0em}         % No paragraph indenting
\setlength{\parskip}{0.5em}         % Paragraph spacing

\newcommand*{\begineasylist}{\begin{easylist}[itemize]\ListProperties(Style*=$\bullet$\quad, Style2*=\tiny$\blacksquare$\quad, Style3*=$\circ$\quad, Style4*=$\diamond$\quad, FinalSpace=1em, Space=0em, Space*=0em)}

\newcommand*{\begineasylistnumbered}{\begin{easylist}[enumerate]\ListProperties(Numbers=a, Space=0em, Space*=0em)}

\begin{document}

\section*{MSCI 261 Midterm Review 2 (Chpt. 6-9)}

\subsection*{Depreciation}
\begineasylist

# \textbf{Market value}: value at which an asset can be sold in a market (usually estimated)

# \textbf{Book value}: value calculated for accounting purposes, using a depreciation model

# \textbf{Scrap value}: value at the end of an asset's \underline{physical} life (when it's broken up for its parts)

# \textbf{Salavage value}: value at the end of an asset's \underline{useful} life (when it's sold)

# \textbf{Straight-line depreciation}: linear diminishment of book value
\[ D_{sl}(n) = \frac{P - S}{N} = \text{amount of depreciation in period $n$} \]
\[ BV_{sl}(n) = P - nD_{sl} = \text{book value at the end of period $n$} \]
## $P = $ current market value/purchase price
## $S = $ salvage value after $N$ periods
## $N = $ \# of periods in useful life

# \textbf{Declining-balance depreciation}: proportional diminishment of book value
\[ D_{db}(n) = d \cdot BV_{db}(n-1) = \text{amount of depreciation in period $n$} \]
\[ BV_{db}(n) = P(1-d)^n = \text{book value at the end of period $n$} \]
\[ d = \sqrt[N]{\frac{S}{P}} = \text{depreciation rate, given $P$, $S$, and $N$} \]

\end{easylist}
\subsection*{Financial Accounting}
\begineasylist

# \textbf{Balance sheet}: snapshot of a firm's financial position at a point in time
## (Current assets + long-term assets) = (Current liabilities + long-term liabilities) + (Owner's equity)

# \textbf{Income statement}: summary of a firm's revenues and expenses over an accounting period
## Income before taxes = Revenues $-$ expenses
## Net income = Income before taxes $-$ taxes

# Liquidity ratios: ability of a firm to meet its current liability obligations
## \textbf{Working capital} = Current assets - Current liabilities
## \textbf{Current ratio/Working capital ratio} = $\dfrac{\text{Current assets}}{\text{Current liabilities}}$
## \textbf{Acid-test ratio/quick ratio} = $\dfrac{\text{Quick assets}}{\text{Current liabilities}}$
### Quick assets = Current assets $-$ Inventories $-$ Prepaid items

# Leverage/debt-management ratios: how much a firm relies on debt for its operations
## \textbf{Equity ratio} = $\dfrac{\text{Owner's equity}}{\text{Total assets}}$

# Efficiency/asset-management ratios: how efficiently a firm uses its assets
## \textbf{Inventory-turnover ratio} = $\dfrac{\text{Sales}}{\text{Inventories}}$

# Profitability ratios: how productively a firm employs its assets to produce profit
## \textbf{Return-on-assets ratio} = $\dfrac{\text{Net income (before extraordinary items)}}{\text{Total assets}}$
## \textbf{Return-on-equity ratio} = $\dfrac{\text{Net income (before extraordinary items)}}{\text{Total equity}}$


\end{easylist}
\subsection*{S}
\begineasylist



\end{easylist}

\end{document}