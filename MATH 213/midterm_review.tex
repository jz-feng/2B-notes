\documentclass[11pt, oneside]{article}
\usepackage[margin=1in]{geometry}
\geometry{letterpaper}
\usepackage{amssymb}
\usepackage[fleqn]{amsmath}
\usepackage[sharp]{easylist}
\usepackage{relsize}

\pagenumbering{gobble}              % No page numbering
\setlength{\parindent}{0em}         % No paragraph indenting
\setlength{\parskip}{0.5em}         % Paragraph spacing

\newcommand*{\begineasylist}{\begin{easylist}[itemize]\ListProperties(Style*=$\bullet$\quad, Style2*=\tiny$\blacksquare$\quad, Style3*=$\circ$\quad, Style4*=$\diamond$\quad, FinalSpace=1em, Space=0em, Space*=0em)}

\newcommand*{\begineasylistnumbered}{\begin{easylist}[enumerate]\ListProperties(Numbers=a, Space=0em, Space*=0em)}

\begin{document}

\section*{MATH 213 (Important Results)}

\subsection*{Separable ODE}
\begineasylist

# \textbf{Goal}: separate functions and derivatives of $x$ and $y$ to either side of the equation
\begin{align*}
f(x) &= g(y)y'\\
\implies \int f(x)dx &= \int g(y)dy \qquad \text{then integrate both sides as normal}
\end{align*}

\end{easylist}
\subsection*{Exact ODE}
\begineasylist
\begin{align*}
M(x,y)dx + N(x,y)dy &= 0\\
&= du \quad \text{where $u$ is some function of $x$ and $y$}
\end{align*}
# \textbf{Goal}: find function $u(x,y) = C$ (aka. an implicit solution of $y$)
# The equation is \textbf{exact} if \& only if $\dfrac{\partial M}{\partial y} = \dfrac{\partial N}{\partial x}$
\begin{align*}
u &= \int Mdx + k(y)\\
\text{then set} \quad N &= \frac{\partial u}{\partial y} \quad \text{to solve for $k(y)$}
\end{align*}

# Alternatively,
\begin{align*}
u &= \int Ndy + k(x)\\
\text{then set} \quad M &= \frac{\partial u}{\partial x} \quad \text{to solve for $k(x)$}
\end{align*}

# If equation is \textbf{not exact}, find the \textbf{integrating factor} $\mu$ such that:
\[ \frac{\partial}{\partial y} \mu M = \frac{\partial}{\partial x} \mu N \]
\[ \text{then solve as} \quad \mu M(x,y)dx + \mu N(x,y)dy = 0 \]

\end{easylist}
\subsection*{First-Order Linear ODE (With Variable Coefficients)}
\begineasylist

# \textbf{Homogeneous}: $y' + p(x)y = 0$
\[ y(x) = Ce^{-h}, \quad h = \int p(x)dx \]

# \textbf{Nonhomogeneous}: $y' + p(x)y = q(x)$
\[ y(x) = e^{-h}(\int e^hq(x)dx + C), \quad h = \int p(x)dx \]

\end{easylist}
\subsection*{Nth-Order Linear ODE (Homogeneous)}
\begineasylist
\begin{align*}
y^{(n)} + p_1(x)y^{(n-1)} + \ldots + p_{n-1}(x)y' + p_{n}y &= r(x)\\
&= L[y] \quad \text{(\underline{differential operator})}
\end{align*}
# \textbf{General solution}:
\begin{align*}
&y = c_1y_1 + \ldots + c_ny_n\\
&\text{where } y_1 \ldots y_n \text{ are linearly independent \underline{particular solutions}}
\end{align*}

\end{easylist}
\subsection*{Nth-Order Linear ODE (Homogeneous w/ Constant Coefficients)}
\begineasylist

# \textbf{Characteristic equation} of $L[y] = 0$ is:
\[ \lambda^n + a_{n-1}\lambda^{n-1} + \ldots + a_1\lambda + a_0 = 0 \qquad \text{with roots } \lambda_1 \ldots \lambda_n \]

# \textbf{General solution}:
\begin{align*}
&y = c_1e^{\lambda_1x} + \ldots + c_ne^{\lambda_nx}\\
&\text{where every } y = e^{\lambda x} \text{ is linearly independent if every } \lambda \text{ is distinct}
\end{align*}

# \textbf{Repeated roots}: for a root $\lambda$ of order $k$, 
\begin{align*}
e^{\lambda x}, xe^{\lambda x}, x^2e^{\lambda x} \ldots x^{k-1}e^{\lambda x} \qquad \text{are linearly independent solutions}
\end{align*}

# \textbf{Complex roots}: for two complex roots $\lambda_1 = ik, \lambda_2 = -ik$,
\begin{align*}
y = c_1e^{ikx} + c_1e^{-ikx} = A\cos(kx) + B\sin(kx)
\end{align*}

\end{easylist}
\subsection*{Nth-Order Linear ODE (Nonhomogeneous)}
\begineasylist
\[ L[y] = f_1(x) + \ldots + f_k(x) \]

# \textbf{General solution}:
\begin{align*}
&y = y_h + y_{p1} + \ldots + y_{pk}\\
\\
&\text{where $y_h$ is a \underline{general solution} of $L[y] = 0$}\\
&\text{and $y_{pi}$ is a \underline{particular solution} of $L[y] = f_i(x)$}
\end{align*}

# \textbf{Method of undetermined coefficients}:
## $f_i(x) \quad \rightarrow \quad y_pi(x) = $ sum of \underline{linear independent derivatives}
## $f_i(x) = e^{kx} \quad \rightarrow \quad y_pi(x) = Ce^{kx}$
## $f_i(x) = x^n, n \geq 0 \quad \rightarrow \quad y_pi(x) = C_nx^n + \ldots + C_1x + C_0$
## $f_i(x) = \cos(kx), \sin(kx) \quad \rightarrow \quad y_pi(x) = A\cos(kx) + B\sin(kx)$
## $f_i(x) = e^{rx}\cos(kx), e^{rx}\sin(kx) \quad \rightarrow \quad y_pi(x) = e^{rx}A\cos(kx) + B\sin(kx)$
## Substitute $y_p$ into LHS and match coefficients with RHS:
\[ L[y_p] = f(x) \qquad \text{and solve for constants} \]


\end{easylist}
\subsection*{Laplace Transform}
\begineasylist

\[ F(s) = L\{f\}(t) = \int_0^\infty f(t)e^{-st}dt \]

# \textbf{Transforms of derivatives}:
\[ L\{f'\} = sF - f(0) \]
\[ L\{f^{(n)}\} = s^nF - s^{n-1}f(0) - \ldots - sf^{(n-2)}(0) - f^{(n-1)}(0) \]

# \textbf{Transforms of integrals}:
\[ L\{ \int_0^t f(r)dr \} = \frac{F}{s} \]

# \textbf{Derivatives of transforms} (multiplication):
\[ L\{tf(t)\} = -F'(s) \]

# \textbf{Integrals of transforms} (division):
\[ L\{\frac{f(t)}{t}\} = \int_s^\infty F(r)dr \]

# \textbf{S-shifting}:
\[ L\{ e^{at}f(t) \} = F(s-a) \]

# \textbf{T-shifting}:
\[ L\{f(t-a)H(t-a)\} = e^{-as}F(s) \]

# \textbf{Dirac's delta function}:
\[ L\{ \delta(t-a)f(t) \} = e^{-as}f(a) \]

# \textbf{Periodic functions}: if $f(t + T) = f(t)$ for all $t$ in domain,
\[ L\{f(t)\} = \frac{1}{1-e^{-sT}}\int_0^T f(t)e^{-st}dt \]

# \textbf{Convolution}:
\[ F(s)G(s) = L\{f(t) * g(t)\} = L\{\int_0^t f(r)g(t-r)dr\} \]

# \textbf{Initial value theorem}: if $f$ is continuous, $f'$ is piecewise continuous, and $f, f'$ are of exponential order,
\[ \lim_{s\rightarrow\infty} sF(s) = f(0) \]

\end{easylist}

\end{document}